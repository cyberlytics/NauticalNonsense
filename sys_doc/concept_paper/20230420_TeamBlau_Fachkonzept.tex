\documentclass[a4paper, 10pt, conference]{IEEEtran}

\IEEEoverridecommandlockouts

\usepackage[utf8]{inputenc}
\usepackage[T1]{fontenc}
\usepackage[colorlinks=true,linkcolor=black,anchorcolor=black,citecolor=black,filecolor=black,menucolor=black,runcolor=black,urlcolor=black]{hyperref}
\usepackage{graphicx}
\usepackage[ngerman]{babel}
\usepackage[style=ieee]{biblatex}

\addbibresource{references.bib}

\graphicspath{ {./images/} }

\begin{document}

\makeatletter
\newcommand{\linebreakand}{%
\end{@IEEEauthorhalign}
\hfill\mbox{}\par
\mbox{}\hfill\begin{@IEEEauthorhalign}
}
\makeatother

\title{\LARGE \bf
Konzeptpapier: Battleship
}

\author{

\IEEEauthorblockN{Jakob Götz} 
\IEEEauthorblockA{\textit{j.goetz@oth-aw.de}}\and
\IEEEauthorblockN{Uwe Kölbel} 
\IEEEauthorblockA{\textit{u.koelbel@oth-aw.de}}\and
\IEEEauthorblockN{Maximilian Schlosser} \IEEEauthorblockA{\textit{m.schlosser@oth-aw.de}}\and
\IEEEauthorblockN{Oliver Schmidts} \IEEEauthorblockA{\textit{o.schmidts@oth-aw.de}}\linebreakand
\IEEEauthorblockN{Jan Schuster} 
\IEEEauthorblockA{\textit{j.schuster@oth-aw.de}}\and
\IEEEauthorblockN{Philipp Seufert} 
\IEEEauthorblockA{\textit{p.seufert@oth-aw.de}}\and
\IEEEauthorblockN{Fabian Wagner} 
\IEEEauthorblockA{\textit{f.wagner@oth-aw.de}}
}

\maketitle
\thispagestyle{empty}
\pagestyle{empty}

%TODO hier noch in unser Projekt umändern
\begin{abstract}
    Wir beschäftigen uns mit der Speicherung und Suche vernetzter Informationen von Videospielen.
    Dazu stellen wir \textit{SGDb} vor -- eine webbasierte Anwendung mit einer Graphen-basierten Suche von Videospielen.
\end{abstract}


\section{Einleitung}\label{sec:einleitung}

Spiele, die im Browser ausführbar sind, erfreuten sich während der Corona-Pandemie größter Beliebtheit. Die Menschen durften sich aufgrund der hohen Ansteckungsrate des Virus nicht in der Realität treffen. 

Dies war der Grund, weshalb diese regelmäßigen Treffen häufig in die digitale Form wechselten. Es wurden Spielabende mit beispielsweise Skribbl.io \cite{skribblio} veranstaltet, da für die Teilnahme keine leistungsfähigen Rechner, sondern meist lediglich ein Smartphone oder Ähnliches benötigt wurde.

Das Brettspiel \glqq Schiffe versenken!\grqq{} ist ein sehr beliebtes und intuitives Spiel, das seit Jahrzehnten Generationen von Spielern begeistert. Dieses Spiel bietet eine perfekte Mischung aus Strategie, Glück und Unterhaltung, die es zu einem zeitlosen Klassiker gemacht hat.

Aus diesem Grund wollen wir in dieser Modularbeit das Spiel im Browser implementieren, damit unabhängig von äußeren Faktoren wie beispielsweise der Corona-Pandemie, das Gesellschaftsspiel Battleship gespielt werden kann. 

\section{Verwandte Arbeiten}\label{sec:verwandte_arbeiten}

Im Jahr 2015 veröffentlichte Matheus Valadares ein simples Browserspiel namens Agar.io \cite{agario}. Das Spielprinzip ist einfach: Man steuert eine Zelle in einer Petrischale und versucht größer zu werden, indem man Agar-Pellets und kleinere Zellen verschluckt. Das Spiel erfreute sich großer Beliebtheit und die Zahl der monatlichen Spieler ist schnell in die Millionen gestiegen \cite{takahashi2017}. Um einen solch rapiden Anstieg an Spielern ohne Ausfälle oder Überlastungen bewältigen zu können, ist eine Cloud-native Architektur ideal. Dazu wird typischerweise die Anzahl der aktiven Server an die Anzahl der aktiven Spieler angepasst. Ein solcher Skalierungsmechanismus ist auch für unser Spiel vorgesehen.

Agar.io bietet mittlerweile eine Vielzahl von Features, die für Online-Spiele typisch sind. Dazu gehören verschiedene Spielmodi, eine Login-Funktion, Mikrotransaktionen und Skins. Es ist denkbar, jedes dieser Features auch in unser Spiel zu integrieren.

\section{Anforderungen}\label{sec:anforderungen}
%TODO Stimmt das mit den drei primären Komponenten noch?
In der Anforderungsanalyse wurden drei primäre Komponenten identifiziert, die es umzusetzen gilt: (1) Die Menüführung (2) das Gameplay und (3) das Dashboard. Als Stakeholder wurden Spieler und Entwickler identifiziert. Die Anforderungen werden im Folgenden als User Stories beschrieben.


\subsection{Menüführung}
Als Spieler möchte ich ein intuitives Menü haben, in dem ich folgende Funktionen auswählen kann:
\begin{itemize}
	\item Meinen Spielernamen eingeben
	\item Ein Spiel starten
	\item Eine Anleitung aufrufen
	\item Eine Bestenliste einsehen
	\item Ein Button zum Ein- und Ausschalten der Musik
\end{itemize}

\subsection{Spielmodi}
Als Spieler möchte ich zwischen verschiedenen Spielmodi wählen können:
\begin{itemize}
	\item Gegen Computer
	\item Gegen zufälligen Gegner
	\item Gegen Freund
\end{itemize}

\subsection{Spielbeginn}
Als Spieler möchte ich das Spiel nach folgenden Regeln vorbereiten:
\begin{itemize}
	\item Jeder Spieler eine feste Anzahl an spezifischen Schiffen, welche auf dem Spielfeld platziert werden
	\item Schiffe dürfen sich bei der Platzierung nicht überschneiden
	\item Sind alle Schiffe platziert, kann der Spieler seine Spielbereitschaft über einen Button signalisieren
\end{itemize}

\subsection{Spielbrett}
Als Spieler möchte aktuelle Informationen über den Spielverlauf auf dem Spielbrett angezeigt bekommen:
\begin{itemize}
	\item Die eigenen Schiffe werden aufgedeckt angezeigt, die gegnerischen Schiffe sind verdeckt
	\item Treffer und verfehlte Schüsse sind auf dem Spielbrett markiert
	\item (optional) Ein Chat mit dem Gegner angezeigt
\end{itemize}

\subsection{Aktive Phase}
Als Spieler möchte ich in der aktiven Phase des Spiels:
\begin{itemize}
	\item Ein Feld auf dem Spielbrett markieren können, welches beschossen werden soll
	\item Ein Button haben, um den Zug zu beenden
\end{itemize}

%TODO Erreichbarkeit und Verzögerung voneinander trennen? Sind beides recht große Topics für sich
\subsection{Erreichbarkeit}
Als Spieler möchte ich:
\begin{itemize}
	\item Unabhängig von der aktuellen Spielerzahl spielen können
	\item Keine Verzögerung zwischen den Zügen wahrnehmen
\end{itemize}

%TODO Hier nochmal aufteilen? Bestenliste in Menü, Detailansicht direkt nach Spiel? Weitere Statistik-Reiter in Menü für sowas wie Hit-Rate usw?
\subsection{(Optional) Detailansicht}
Als Spieler möchte ich ein Detailansicht, dass Informationen über vergangene Spiele darstellt. Denkbar sind:
\begin{itemize}
	\item Eine Bestenliste
	\item Anzahl der benötigten Schüsse bis zum Sieg
	\item Beliebteste Schiffspositionen
	\item Gewinnrate gegen Computer
\end{itemize}

\subsection{Testabdeckung}
Als Entwickler möchte ich eine ausreichende Testabdeckung, um Fehler frühzeitig zu erkennen.
\begin{itemize}
	\item Frontend-Code Abdeckung liegt bei mindestens 50%
	\item Backend-Code Abdeckung liegt bei mindestens 50%
	\item Für die Testabdeckung werden jeweils geeignete Frameworks gewählt
\end{itemize}

\subsection{Dashboard}
Als Entwickler möchte ein Dashboard, welches mir Informationen zur Infrastruktur und der Nutzung anzeigt.
\begin{itemize}
	\item Wahl eines geeigneten Tools zur Darstellung der Informationen
	\item Die Informationen im Dashboard müssen aktuell sein
	\item Die Oberfläche des Dashboards muss intuitiv zu bedienen sein
\end{itemize}

\subsection{Cloud Native}
Als Entwickler möchte die Anwendung als Cloud Native bereitstellen.
\begin{itemize}
	\item Es wird eine Container-Technologie eingesetzt
	\item Es kommt ein Orchestrierungs-Tool zum Einsatz
\end{itemize}

\subsection{(Optional) Bereitstellung in der Cloud}
Als Entwickler möchte die Anwendung über einen Cloud-Anbieter bereitstellen.
\begin{itemize}
	\item Die Kosten hierfür müssen im Maße bleiben
	\item Die Anwendung ist über das Internet zu erreichen
\end{itemize}

\section{Methoden}\label{sec:methoden}

Das Projekt setzt sich aus drei Komponenten zusammen: Frontend, Backend und Datenbank.
Die einzelnen Komponenten werden jeweils in einem Docker-Container ausgeliefert.
Sie werden mit einem in der Cloud gehosteten Kubernetes-Cluster orchestriert und erreichbar gemacht.

Für die Visualisierung des Spielgeschehens im Frontend wird das Game Framework Phaser mit JavaScript verwendet. 
Ein Webserver stellt das Frontend für den Nutzer zur Verfügung.
Das Backend wird in Python implementiert und stellt eine RESTful-API sowie die Spiellogik zur Verfügung.
Diese wird mithilfe des Frameworks FastAPI realisiert und dient zur Kommunikation mit dem Frontend.
Als Datenbank wird MongoDB verwendet. Die Spielverläufe werden als JSON-Objekte serialisiert und für spätere Analysen gespeichert.

\printbibliography

\end{document}
